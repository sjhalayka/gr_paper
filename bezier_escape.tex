\documentclass[12pt]{article}

\title{Numerical general relativity using Euler integration}
\author{S. Halayka\footnote{sjhalayka@gmail.com}}
\date{\today\;\currenttime}

\usepackage{datetime}
\usepackage{listings}
\usepackage{cite}
\usepackage{xcolor}
\usepackage{graphicx}
\usepackage{setspace}
\usepackage{amsmath}
\usepackage{url}
\usepackage[margin=0.9in]{geometry}
%\doublespace



\begin{document}



 
\maketitle

\begin{abstract}
Two numerical solutions for general relativity are provided.
The topics under consideration are the deflection of light and neutrinos by the Sun, and the relativistic rotation of Mercury's orbit plane.
To simplify the calculations, Euler integration is used.
\end{abstract}





\section{Introduction}

- Used Euler

- Shapiro time delay
- Einstein-Pound-Rebka red/blueshift







Euler is just fine for calculating deflection and precession.




- The predicted amount is $1.75$ arc seconds, which is practically identical to the amount given by the analytical solution.




The analytical solution for the deflection angle is:
\begin{equation}
\delta_{Deflection} = \frac{4GM_{Sun}}{c^2} \left( \frac{1}{\pi \times 180 \times 3600} \right) = 1.75 \textrm{ arc seconds}.
\end{equation}

The analytical solution for the precession of the perihelion of Mercury is:
\begin{equation}
\delta_{Precession} = \frac{6 \pi GM_{Sun}}{c^2 (1 - e^2) a} \left( \frac{1}{ \pi \times 180 \times 3600} \right) \left( \frac{365}{88} \times 100 \right) = 42.937 \textrm{ arc seconds per Earth century},
\end{equation}
where $e = 0.2056$ is the eccentricity and $a = 57.909 \times 10^9$ is the semi-major axis.






Euler leads to a negative rotation, but given a small enough timeslice, this negative rotation becomes negligible:
\begin{equation}
\delta_{Timeslice} = \frac{c}{\lvert\lvert v_{Orbiter} \rvert \rvert} \times 10^{-5}.
\end{equation}




- Altogether the relativistic precession predicted is $46$ arc seconds per Earth century, where the amount given by the analytical solution is $43$ arc seconds per Earth century.

Where $\ell$ denotes location, and $\vec{d}$ denotes the direction vector that points from Mercury toward the Sun:
\begin{equation}
\vec{d} = \ell_{Sun} - \ell_{Orbiter},	
\end{equation}
\begin{equation}
\hat{d} = \frac{\vec{d}}{\lvert\lvert \vec{d} \rvert\rvert}.
\end{equation}
The acceleration is:
\begin{equation}
\vec{g} =  \frac{\hat{d} G M_{Sun}}{{\lvert\lvert \vec{d} \rvert\rvert}^2}.
\end{equation}

\begin{equation}
R_{Schwarzschild} = \frac{2GM_{Sun}}{c^2}.
\end{equation}


Two important values are:
\begin{equation}
\alpha = 2 - \sqrt{1 - \frac{\lvert\lvert v_{Orbiter}\rvert\rvert^2}{c^2}}.
\end{equation}
\begin{equation}
\beta = \sqrt{1 - \frac{R_{Schwarzschild}}{\lvert \lvert \vec{d} \rvert \rvert}}.
\end{equation}

The Euler integration is:
\begin{equation}
v_{Orbiter} = v_{Orbiter} + \alpha \vec{g} \delta_{Timeslice},
\end{equation}
\begin{equation}
L_{Orbiter} = L_{Orbiter} + \beta v_{Orbiter} \delta_{Timeslice}.
\end{equation}






\begin{thebibliography}{9}


\bibitem{misner} Misner et al. Gravitation. (1970)






\end{thebibliography}





\end{document}









