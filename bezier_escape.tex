\documentclass[12pt]{article}

\title{On simulating post-Newtonian gravitation using Euler integration}
\author{S. Halayka\footnote{sjhalayka@gmail.com}}
\date{\today\;\currenttime}

\usepackage{datetime}
\usepackage{listings}
\usepackage{cite}
\usepackage{xcolor}
\usepackage{graphicx}
\usepackage{setspace}
\usepackage{amsmath}
\usepackage{url}
\usepackage[margin=1.0in]{geometry}

%\doublespace

%\usepackage[]{lineno}
%\linenumbers


\begin{document}



 
\maketitle

\begin{abstract}
A simple numerical solution for general relativity in the Newtonian weak field regime is provided.
The topics under consideration are the deflection of photons and neutrinos by the Sun, the relativistic rotation of Mercury's orbit plane, and Shapiro time delay.
To simplify the calculations, Euler integration is used.
Hopefully, this numerical solution can be found to be an important tool, primarily due to this simplicity.
\end{abstract}


\section{Introduction}

In this paper we will describe a numerical solution for the deflection of light and neutrinos by the Sun (e.g. where $v \approx c$), for the relativistic orbit of Mercury around the Sun (e.g. where $v \ll c$), and for the Shapiro time delay of photons.

No tensor calculus is involved -- only high school Physics is required (e.g. Newtonian gravity, along with its use of 3-dimensional vectors).

It is also recommended that one has already read about special relativity \cite{einstein, morin}, and thus kinematic time dilation.
For instance, Einstein's light clock thought experiment is helpful for understanding kinematic time dilation.
After one understands kinematic time dilation, it is not too far a stretch for them to understand gravitational time dilation -- gravitation is the interruption of internal process by some other process at some distance away.

Note that this solution is only valid in the weak field, where the gradient of the gravitational time dilation practically vanishes, and Newton's inverse square law holds true.

Because of its simplicity, this numerical solution can serve as a stepping stone for further education on the subject -- including tensor calculus.







\section{On integration using steps in time}

Here we take into consideration the kinematic and gravitational time dilation / length contraction to calculate the deflection of photons and neutrinos by the Sun, the relativistic rotation of Mercury's orbit plane, and the Shapiro time delay.
We rely on the Schwarzschild solution, which means that the Sun is taken to be spherically symmetric, stationary, static, and non-rotating \cite{einstein, misner, schutz, mcmahon}:
\begin{equation}
\label{efe}
R_{\mu\nu} - \frac{1}{2} R g_{\mu\nu} + \Lambda g_{\mu\nu} = \frac{8\pi G}{c^4} T_{\mu\nu},
\end{equation}
\begin{equation}
\label{schwarzschild_line_element}
ds^2 = -\left( 1 - \frac{R_{s}}{r} \right) c^2 dt^2 + \frac{dr^2}{\left( 1 - \frac{R_{s}}{r} \right)} + r^2 (d\theta^2 + \sin^2 \theta d\phi^2),
\end{equation}
where $R_{s}$ is the Schwarzschild radius
\begin{equation}
\label{schwarzschild_radius}
R_{s} = \frac{2GM}{c^2},
\end{equation}
and the Sun's mass is $M = 1.98847 \times 10^{30}$, Newton's constant is $G = 6.6743 \times 10^{-11}$, and the speed of light in vacuum is precisely $c = 299792458$.

From here on in, we shall use Euclidean 3-dimensional space, with Cartesian coordinates.

In the case of deflection of photons and neutrinos by the Sun, the timeslice that we used is:
\begin{equation}
\label{dt_1}
\delta_{t} = 1.
\end{equation}
The analytical solution for the deflection angle of photons or neutrinos that are just grazing the Sun is:
\begin{equation}
\label{delta_d}
\delta_{d} = \frac{4GM}{c^2 r} \left( \frac{1}{\pi \times 180 \times 3600} \right) = 1.75
\end{equation}
arc seconds, where $r = 6.9634 \times 10^8$ is the distance of closest approach (e.g. the Sun's radius).
The numerically predicted amount is like $1.75$ arc seconds, which is practically identical to the amount given by the analytical solution.

In the case of the precession of the perihelion of Mercury, do note that Euler integration automatically leads to a negative rotation of Mercury's orbit plane, but given a small enough timeslice, this negative rotation becomes negligible.
For instance, where $\vec{v}_{o}$ denotes the orbiter's velocity vector:
\begin{equation}
\label{dt_other}
\delta_{t} = \frac{c}{\lvert\lvert \vec{v}_{o} \rvert \rvert} \times 10^{-5}.
\end{equation}
The analytical solution for the precession of the perihelion of Mercury is:
\begin{equation}
\label{delta_p}
\delta_{p} = \frac{6 \pi G M}{c^2 (1 - e^2) a} \left( \frac{1}{ \pi \times 180 \times 3600} \right) \left( \frac{365}{88} \times 100 \right) = 42.937
\end{equation}
arc seconds per Earth century, where $e = 0.2056$ is the eccentricity and $a = 5.7909 \times 10^{10}$ is the semi-major axis.
The numerically predicted amount is like $43$ arc seconds per Earth century, which is practically identical to the amount given by the analytical solution.





In the case of Shapiro time delay, the timeslice that we used is:
\begin{equation}
\label{dt_1_div_c}
\delta_{t} = \frac{1}{c}.
\end{equation}
The analytical solution for the round-trip Shapiro time delay is:
\begin{equation}
\label{delta_s}
\delta_{s} = \frac{2GM}{c^3} \log\left( \frac{R_x + R_y}{R_x - R_y} \right) = 10^{-4}
\end{equation}
seconds, where 
\begin{equation}
\label{r_x}
R_x = \lvert\lvert (-1000 \times c, R_g, 0) - (0, 0, 0) \rvert\rvert
\end{equation}
is the distance from the sender to the Sun's centre, and
\begin{equation}
\label{r_y}
R_y = \lvert\lvert (-1000 \times c, R_g, 0) - (0, R_g, 0) \rvert\rvert.
\end{equation}
is the distance from the sender to the Sun's surface, which is just grazed at the Sun's radius of $R_g = 6.9634 \times 10^8$.
We assume that the communication system is symmetric, and so the distance from the receiver to the Sun's centre is also $R_x$ and the distance from the receiver to the Sun's grazed surface is also $R_y$.
The numerically predicted amount is like $10^{-4}$ seconds, which is practically identical to the amount given by the analytical solution.
Note that the analytical solution is only an approximation, and that the numerical solution is the correct measure (e.g. the ground truth).

Where $\ell_s$ denotes the Sun's location, $\ell_o$ denotes the orbiter's location, and $\vec{d}$ denotes the direction vector that points from the orbiter toward the Sun:
\begin{equation}
\label{direction_vector}
\vec{d} = \ell_{s} - \ell_{o},	
\end{equation}
\begin{equation}
\label{direction_unit_vector}
\hat{d} = \frac{\vec{d}}{\lvert\lvert \vec{d} \rvert\rvert}.
\end{equation}
The Newtonian acceleration vector is:
\begin{equation}
\label{newton}
\vec{g}_n = \frac{\hat{d} G M}{{\lvert\lvert \vec{d} \rvert\rvert}^2}.
\end{equation}

One important value is closely related to the kinematic time dilation:
\begin{equation}
\label{eq_kinematic}
\alpha = 2 - \sqrt{1 - \frac{\lvert\lvert \vec{v}_{o}\rvert\rvert^2}{c^2}}.
\end{equation}
Another important value is the gravitational time dilation:
\begin{equation}
\label{eq_gravitational}
\beta = \sqrt{1 - \frac{R_{s}}{\lvert \lvert \vec{d} \rvert \rvert}}.
\end{equation}

Finally, the semi-implicit Euler integration is:
\begin{align}
\label{eq_velocity}
\vec{v}_{o}(t + \delta_t) &= \vec{v}_{o}(t) + \delta_{t} \alpha \vec{g}_n, \\
\label{eq_position}
\ell_{o}(t + \delta_t) &= \ell_{o}(t) + \delta_{t} \beta \vec{v}_{o}(t + \delta_t).
\end{align}

Note that Newtonian gravity is the result where $\alpha = \beta = 1$.





\section {Review}

\begin{enumerate}

\item
The Einstein field equations for gravitation are given in Eq. \ref{efe}, which are solved for by Schwarzschild's line element in Eqs. \ref{schwarzschild_line_element} and \ref{schwarzschild_radius}.

\item
Where velocity is equal to the speed of light (or very close, like for neutrinos), we found that in Eq. \ref{dt_1} that a timeslice of 1 second is suitable.

\item
We found that Eq. \ref{delta_d} and the numerical solution both predict the same amount (e.g. $1.75$ arc seconds).

\item
Where velocity is variable, we found that in Eq. \ref{dt_other} that a variable timeslice is suitable.

\item
We found that Eq. \ref{delta_p} and the numerical solution both predict the same amount (e.g. $43$ arc seconds per Earth century).

\item
Where velocity is equal to the speed of light for Shapiro delay, we found that in Eq. \ref{dt_1_div_c} that a timeslice of $1/c$ seconds is suitable.

\item 
We found that Eqs. \ref{delta_s}, \ref{r_x} and \ref{r_y} and the numerical solution both predict the same amount (e.g. $10^{-4}$ seconds).




\item
We found the direction vector and its unit length version in Eqs. \ref{direction_vector} and \ref{direction_unit_vector}.

\item
We found the Newtonian acceleration in Eq. \ref{newton}.

\item
Eq. \ref{eq_kinematic} goes to show that internal process is equal to a resistance to gravitation, and that the effects of gravity are stronger the faster one goes.
It also goes to show that $2G$ is the fundamental constant, not $G$.

\item
Eq. \ref{eq_gravitational} goes to show that internal process is overcome by gravitational time dilation, and that the effects of gravity are stronger the closer one gets to the gravitating body.

\item
Eq. \ref{eq_velocity} goes to show that part of the relativistic precession is due to the kinematic time dilation (e.g where $1 \leq \alpha \leq 2$). 
Mercury is gravitated more than it would be using Newtonian gravitation alone, for Newtonian gravity alone does not take velocity into account.

\item
Eq. \ref{eq_position} goes to show that the planet Mercury dwells longer when it's closer to the Sun (e.g. where $0 \leq \beta < 1$), causing the rest of the relativistic perihelion precession. 
Mercury is gravitated more than it would be using Newtonian gravitation alone, for Newtonian gravity alone does not take the contraction of space around the gravitating body into account.



\end{enumerate}


\section{Conclusion}

The numerical solution to Einstein's field equations that is provided here requires a mastery of only high school-level mathematics and software development.
This allows the student to become gently acquainted with general relativity, before having to master the required tensor calculus.

The solution calculated the deflection of photons and neutrinos by the Sun, the relativistic orbit of Mercury, and Shapiro time delay.
For instance, the solution calculated the deflection angle of $1.75$ arc seconds.
As well, the solution calculated a precession angle of $43$ arc seconds per Earth century.
Finally, the solution calculated a Shapiro time delay of $10^{-4}$ seconds.
Thus, the solution correctly calculates all four Solar System tests of general relativity (including the Pound-Rebka gravitational redshift, by default).







\begin{thebibliography}{9}


\bibitem{einstein} Einstein. Relativity -- The Special and General Theory. (1916)
\bibitem{morin} Morin. Special Relativity for the Enthusiastic Beginner. (2017)

\bibitem{misner} Misner et al. Gravitation. (1970)
\bibitem{schutz} Schutz. A First Course in General Relativity. (1985)
\bibitem{mcmahon} McMahon et al. Relativity Demystified. (2005)

\end{thebibliography}





\end{document}









