\documentclass[12pt]{article}

\title{Post-Newtonian numerical general relativity using Euler integration}
\author{S. Halayka\footnote{sjhalayka@gmail.com}}
\date{\today\;\currenttime}

\usepackage{datetime}
\usepackage{listings}
\usepackage{cite}
\usepackage{xcolor}
\usepackage{graphicx}
\usepackage{setspace}
\usepackage{amsmath}
\usepackage{url}
\usepackage[margin=0.9in]{geometry}
%\doublespace



\begin{document}



 
\maketitle

\begin{abstract}
A numerical solution for general relativity in the Newtonian regime is provided.
The topics under consideration are the deflection of light and neutrinos by the Sun, and the relativistic rotation of Mercury's orbit plane.
To simplify the calculations, Euler integration is used.
\end{abstract}





\section{Introduction}

Here we take into consideration the kinematic and gravitational time dilation / length contraction to calculate the deflection of light and neutrinos by the Sun, and the relativistic rotation of Mercury's orbit plane.

We rely on the Schwarzschild solution, which means that the Sun is taken to be spherically symmetric, stationary, and non-rotating.

In the case of deflection, the timeslice is simply:
\begin{equation}
\delta_{Timeslice} = 1.
\end{equation}
Note that Euler integration automatically leads to a negative rotation in terms of Mercury's orbit plane, but given a small enough timeslice, this negative rotation becomes negligible:
\begin{equation}
\delta_{Timeslice} = \frac{c}{\lvert\lvert \vec{v}_{Orbiter} \rvert \rvert} \times 10^{-5}.
\end{equation}

The analytical solution for the deflection angle is:
\begin{equation}
\delta_{Deflection} = \frac{4GM_{Sun}}{c^2 r} \left( \frac{1}{\pi \times 180 \times 3600} \right) = 1.75 \textrm{ arc seconds},
\end{equation}
where $r = 696340000$ is the distance of closest approach (e.g. the Sun's radius).
The numerically predicted amount is $1.75$ arc seconds, which is practically identical to the amount given by the analytical solution.

The analytical solution for the precession of the perihelion of Mercury is:
\begin{equation}
\delta_{Precession} = \frac{6 \pi GM_{Sun}}{c^2 (1 - e^2) a} \left( \frac{1}{ \pi \times 180 \times 3600} \right) \left( \frac{365}{88} \times 100 \right) = 42.937 \textrm{ arc seconds per Earth century},
\end{equation}
where $e = 0.2056$ is the eccentricity and $a = 57.909 \times 10^9$ is the semi-major axis.
Altogether the numerically relativistic precession predicted amount is like $43$ arc seconds per Earth century, which is practically identical to the amount given by the analytical solution.

Where $\ell$ denotes location, and $\vec{d}$ denotes the direction vector that points from the test particle (e.g. a photon, or a neutrino, or a planet) toward the Sun:
\begin{equation}
\vec{d} = \ell_{Sun} - \ell_{Orbiter},	
\end{equation}
\begin{equation}
\hat{d} = \frac{\vec{d}}{\lvert\lvert \vec{d} \rvert\rvert}.
\end{equation}

The acceleration vector is:
\begin{equation}
\vec{g} =  \frac{\hat{d} G M_{Sun}}{{\lvert\lvert \vec{d} \rvert\rvert}^2}.
\end{equation}

Two important values are:
\begin{equation}
\alpha = 2 - \sqrt{1 - \frac{\lvert\lvert \vec{v}_{Orbiter}\rvert\rvert^2}{c^2}},
\end{equation}
where internal process is seen to be a resistance to gravitation, and
\begin{equation}
\beta = \sqrt{1 - \frac{2GM_{Sun}}{c^2 \lvert \lvert \vec{d} \rvert \rvert}}.
\end{equation}

The Euler integration is:
\begin{equation}
\vec{v}_{Orbiter} = \vec{v}_{Orbiter} + \alpha \vec{g} \delta_{Timeslice},
\end{equation}
\begin{equation}
\ell_{Orbiter} = \ell_{Orbiter} + \beta \vec{v}_{Orbiter} \delta_{Timeslice}.
\end{equation}






\begin{thebibliography}{9}


\bibitem{misner} Misner et al. Gravitation. (1970)






\end{thebibliography}





\end{document}









