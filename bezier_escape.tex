\documentclass[12pt]{article}

\title{On simulating post-Newtonian numerical general relativity using Euler integration}
\author{S. Halayka\footnote{sjhalayka@gmail.com}}
\date{\today\;\currenttime}

\usepackage{datetime}
\usepackage{listings}
\usepackage{cite}
\usepackage{xcolor}
\usepackage{graphicx}
\usepackage{setspace}
\usepackage{amsmath}
\usepackage{url}
\usepackage[margin=0.9in]{geometry}
%\doublespace



\begin{document}



 
\maketitle

\begin{abstract}
A novel numerical solution for general relativity in the Newtonian regime is provided.
The topics under consideration are the deflection of photons and neutrinos by the Sun, and the relativistic rotation of Mercury's orbit plane.
To simplify the calculations, Euler integration is used.
\end{abstract}





\section{On integration using steps in time}

Here we take into consideration the kinematic and gravitational time dilation / length contraction to calculate the deflection of photons and neutrinos by the Sun, and the relativistic rotation of Mercury's orbit plane.
We rely on the Schwarzschild solution, which means that the Sun is taken to be spherically symmetric, stationary, and non-rotating \cite{misner}:
\begin{equation}
R_{\mu\nu} - \frac{1}{2} R g_{\mu\nu} + \Lambda g_{\mu\nu} = \frac{8\pi G}{c^4} T_{\mu\nu},
\end{equation}
\begin{equation}
ds^2 = -\left( 1 - \frac{R_{s}}{r} \right) c^2 dt^2 + \frac{dr^2}{\left( 1 - \frac{R_{s}}{r} \right)} + r^2 (d\theta^2 + \sin^2 \theta d\phi^2),
\end{equation}
where $R_{s}$ is the Schwarzschild radius
\begin{equation}
R_{s} = \frac{2GM}{c^2}.
\end{equation}

In the case of deflection of photons and neutrinos by the Sun, the timeslice that we used is:
\begin{equation}
\delta_{t} = 1.
\end{equation}



The analytical solution for the deflection angle of photons or neutrinos that are just grazing the Sun is:
\begin{equation}
\delta_{d} = \frac{4GM}{c^2 r} \left( \frac{1}{\pi \times 180 \times 3600} \right) = 1.75
\end{equation}
arc seconds, where $r = 696340000$ is the distance of closest approach (e.g. the Sun's radius).
The numerically predicted amount is $1.75$ arc seconds, which is practically identical to the amount given by the analytical solution.

In the case of the precession of the perihelion of Mercury, do note that Euler integration automatically leads to a negative rotation of Mercury's orbit plane, but given a small enough timeslice, this negative rotation becomes negligible.
For instance, where $\vec{v}_{o}$ denotes the orbiter's velocity:
\begin{equation}
\delta_{t} = \frac{c}{\lvert\lvert \vec{v}_{o} \rvert \rvert} \times 10^{-5}.
\end{equation}

The analytical solution for the precession of the perihelion of Mercury is:
\begin{equation}
\delta_{p} = \frac{6 \pi G M}{c^2 (1 - e^2) a} \left( \frac{1}{ \pi \times 180 \times 3600} \right) \left( \frac{365}{88} \times 100 \right) = 42.937
\end{equation}
arc seconds per Earth century, where $e = 0.2056$ is the eccentricity and $a = 57.909 \times 10^9$ is the semi-major axis.
Altogether the numerically predicted amount is like $43$ arc seconds per Earth century, which is practically identical to the amount given by the analytical solution.

Where $\ell_s$ denotes the Sun's location, $\ell_o$ denotes the orbiter's location, and $\vec{d}$ denotes the direction vector that points from the orbiter (e.g. a photon, or a neutrino, or a planet) toward the Sun:
\begin{equation}
\vec{d} = \ell_{s} - \ell_{o},	
\end{equation}
\begin{equation}
\hat{d} = \frac{\vec{d}}{\lvert\lvert \vec{d} \rvert\rvert}.
\end{equation}

The Newtonian acceleration vector is:
\begin{equation}
\vec{g}_n =  \frac{\hat{d} G M}{{\lvert\lvert \vec{d} \rvert\rvert}^2}.
\end{equation}

One important value is:
\begin{equation}
\alpha = 2 - \sqrt{1 - \frac{\lvert\lvert \vec{v}_{o}\rvert\rvert^2}{c^2}}.
\end{equation}

Another important value is:
\begin{equation}
\beta = \sqrt{1 - \frac{R_{s}}{\lvert \lvert \vec{d} \rvert \rvert}}.
\end{equation}

Finally, the Euler integration is:
\begin{equation}
\vec{v}_{o}(t + \delta_t) = \vec{v}_{o}(t) + \alpha \vec{g}_n \delta_{t},
\end{equation}
\begin{equation}
\ell_{o}(t + \delta_t) = \ell_{o}(t) + \beta \vec{v}_{o} \delta_{t}.
\end{equation}

Note that Eq. 11 goes to show that internal process is equal to a resistance to gravitation, and that $2G$ is the fundamental constant, not $G$.





\begin{thebibliography}{9}


\bibitem{misner} Misner et al. Gravitation. (1970)






\end{thebibliography}





\end{document}









